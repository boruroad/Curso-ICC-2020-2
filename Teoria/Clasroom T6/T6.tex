\documentclass[a4paper]{article}
\usepackage[top=2cm,bottom=2cm,left=2.5cm,rigth=2.5cm,heightrounded]{geometry}
\usepackage[spanish]{babel}
\usepackage[utf8]{inputenc}
\usepackage{amsmath}
\usepackage{graphicx}
\usepackage[colorinlistoftodos]{todonotes}
\usepackage{pifont}
\usepackage{ragged2e}
\usepackage[T1]{fontenc}
\usepackage{lmodern}
\usepackage{parskip}
\usepackage{amsmath}
\usepackage{amssymb}
\usepackage{tcolorbox}
\tcbuselibrary{listingsutf8}
\usepackage[hidelinks]{hyperref}


% Definir cuadro de ancho del texto

% OPCIONES
% colback: color de fondo
% colframe: color de borde
% fonttitle: estilo de título
% title: título de la cuadro o referencia a argumento

\title{Introducción a Ciencias de la Computación \\Tarea 6: Herencia y polimorfismo}

\author{\large Bonilla Ruiz Roberto Adrián}

\date{\large Fecha de entrega: 14 de Mayo 2020}

\begin{document}
\maketitle


\newtcolorbox[auto counter,number within=section]{example}[2][]
{colback=white!5!white,colframe=green!75!black,fonttitle=\bfseries, title=~}

\begin{enumerate}


\\ 
%\section{Ejemplos}
\begin{example}
[label={ex:serie}]{ \bigskip\textbf{\qquad \qquad Respuesta=} }
\item\large\textbf{Sopita de letras 
\smallskip} \\ 
{ Encuentra en la sopa de letras las palabras que acompletan las siguientes frases:\\
(Marcalas en la sopa de letra y acompleta las frases)\\
  
}

\includegraphics[width=1.0\textwidth]{abc.png}

\end{example}\medskip
\end{enumerate}

\begin{example}

{1. Las clases Padres también son llamadas ''Clase base'' ó \textit{\underline{Superclase}}.\\

\samallskip 2. Las clases hijas también son llamadas ''Clase derivada'' ó \textit{\underline{Subclase}}.\\

3. La herencia respalda el concepto de \textit{\underline{reutilización}}. Al heredar, estamos reutilizando los campos/atributos y métodos de la clase existente.\\


4. Java soporta la herencia permitiendo una clase a incorporar otra clase en su declaración. Esto se hace mediante el uso de la palabra clave \textit{\underline{extends}}.\\


5. Existen 3 tipos de herencia a través de clases: \textit{\underline{multinivel}},\textit{\underline{ única}} y \textit{\underline{jerárquica}}.\\

6. La clase \textit{\underline{\textbf{Object}}} no tiene superclase, toda clase es implícitamente una subclase de la clase \textit{\underline{Object}}.\\ 

7. Una subclase solo puede tener una super clase. Esto se debe a que Java no admite herencia \textit{\underline{múltiple}} con clases.\\

8. Los \texitf{\underline{constructores}} y métodos/atributos \textit{\underline{privados}} no son heredados a las subclases, pero el costructor de la superclase puede invocarse desde la subclase y acceder a dichos atributos por medio de setter y getters.\\
\newpage
9. El \textit{\underline{poliformismo}} en Java y en la POO se refiere a que un objeto puede tomar diferentes formas de comportarse, es decir, que las subclases de una clase pueden definir su propio comportamiento.\\

10. Que pasa si queremos por ejemplo hacer referencia al método constructor del padre o a algún atributo del padre si tal vez dicho atributo se llama igual que el atributo del hijo, habría una confusión. Para esto está la palabra clave \texit{\underline{super}} con la cual podemos hacer referencia a metodos y atributos de la clase padre.
}
\end{example}






 

\end{document}