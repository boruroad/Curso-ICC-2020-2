\documentclass[a4paper]{article}
\usepackage[spanish]{babel}
\usepackage[utf8]{inputenc}
\usepackage{amsmath}
\usepackage{graphicx}
\usepackage[colorinlistoftodos]{todonotes}
\usepackage{ragged2e}
\usepackage{pifont}
\usepackage[T1]{fontenc}
\usepackage{lmodern}
\usepackage{parskip}
\usepackage{amsmath}
\usepackage{amssymb}
\usepackage{tcolorbox}
\tcbuselibrary{listingsutf8}

% Definir cuadro de ancho del texto

% OPCIONES
% colback: color de fondo
% colframe: color de borde
% fonttitle: estilo de título
% title: título de la cuadro o referencia a argumento

\title{Tarea 3 \\ Introducción a Ciencias de la Computación}

\author{\large Bonilla Ruiz Roberto Adrián}

\date{\today}

\begin{document}
\maketitle


\newtcolorbox[auto counter,number within=section]{example}[2][]
{colback=green!5!white,colframe=green!75!black,fonttitle=\bfseries, title=~}

{\Large\ding{39}{ Preguntas: } \large{(Primera sección) \smallskip} }

\\ {  a) ¿Por que si sabemos que las cadenas, una vez creadas, no pueden ser modificadas, podemos realizar cada una de las operaciones anteriores y en particular la asignacion en la linea 6?}

%proof
%\section{Ejemplos}

\begin{example}
[label={ex:serie}]{ \bigskip\textbf{\qquad \qquad Respuesta=} }
 Porque tras haberse creado el objeto de tipo \texttt{String}, en la linea 6, vemos que la variable puede modificar su estado conforme la ejecución del programa, a menos que desde un principio se declare como constante con la palabra reservada \texttt{final}, esta solo puede cambiar de valor, si desde un inicio también se hace de tipo estatica.
\end{example}\medskip

\\ {(b) ¿Por que las lıneas 9 y 10 regresan el mismo valor si son dos objetos diferentes?}

\begin{example}
[label={ex:serie}]{ \bigskip\textbf{\qquad \qquad Respuesta=} }
 \\Porque \texttt {cadenaOriginal} es estática, lo cual quiere decir que si cambiamos su valor, entonces todos los objetos creados por esa clase tambien cambian.
\end{example}\medskip





\\ {(c) Si en la impresion 9 y 10 da el mismo resultado, ¿Por que en la impresion de la linea 11 y 12 no son iguales?}

\begin{example}
[label={ex:serie}]{ \bigskip\textbf{\qquad \qquad Respuesta=} }
 
Porque como el estado de la cadena  "Soy una cadena del objeto" no es estático,quiere decir que si cambia para el, no afecta a todos.
\end{example}\medskip

\newpage

\\ {(d) Descomenta la lınea 0, debe imprimir en consola un error, ¿Que significa ese error?¿Como se soluciona?}

\begin{example}
[label={ex:serie}]{ \bigskip\textbf{\qquad \qquad Respuesta=} }
El error es por el rango que cada método usa, en el caso del \textit{metodo.lenght()}contamos desde 1, mientras que \textit{charAt()}corrre desde cero).
Aún asi este problema  se corrige con restarle 1 a la longitud de la cadena:\\
		\centerline{\textit{(cadena.length()-1)}}\\
\\De esta forma ambos "parten en cero". 
\end{example}\bigskip




%segunda sección
% Cuadro estrecho
\newtcbox{cuadro}[1]{colback=blue!5!white,colframe=blue!75!black,fonttitle=\bfseries,title=~}


{\Large\ding{74}{ Preguntas: } \large{(Segunda sección)\smallskip} }



{(a) Escribe tres formas de creacion de una cadena:}

\begin{tcolorbox}[colback=blue!5!white,colframe=blue!75!black,fonttitle=\bfseries,title=~]
 \textit\centering{System.out.println("Primer forma de crear una cadena");\\
	String cadena = new String("Segunda forma de crear una cadena");\\
	String nuevacadena = "Tercer forma de crear una cadena";}
\end{tcolorbox}\bigskip 






{(b) ¿Cual es el valor por defecto de un objeto \texttt String?}
\begin{tcolorbox}[colback=blue!5!white,colframe=blue!75!black,fonttitle=\bfseries,title=~]
 \textit\centering{null}
\end{tcolorbox}\bigskip 


\\ { \qquad ¿Y el valor por defecto de un \texttt char?}
\begin{tcolorbox}[colback=blue!5!white,colframe=blue!75!black]
 \textit\centering{'u0000'}
\end{tcolorbox}\bigskip

\\ {(c) Escribe dos formas de concatenar cadenas en java:}
\begin{tcolorbox}[colback=blue!5!white,colframe=blue!75!black,fonttitle=\bfseries,title=~]

	\item{La primer forma es la más sencilla (A través de una impresión).}\smallskip

	\begin{center}\textit{System.out.println("Primer forma " \ + "de concatenar cadenas");}\medskip \end{center}
\\
	{La segunda forma requiere crear nuestras cadenás y después imprimirlas con el operador de concatenación.}\smallskip

	\centering\textit{String oracion1 = "Segunda forma "; 
	\\String oracion2 =" de concatenar cadenas";\\
	System.out.println(oracion1 + oracion2);}

\end{tcolorbox}\bigskip 
\newpage





\\{(d) ¿Que es ”CAST” en java?}
\begin{tcolorbox}[colback=blue!5!white,colframe=blue!75!black,fonttitle=\bfseries,title=~]
{Es un procedimiento para transformar una variable primitiva de un tipo a otro.}
\end{tcolorbox}\bigskip 


 \\{(e) ¿Que hace la siguiente lınea de codigo?\\ }

 \centerline{\hspace{1cm}\texttt{int i = Integer.parseInt(myString);}\smallskip}\\
 
\begin{tcolorbox}[colback=blue!5!white,colframe=blue!75!black,fonttitle=\bfseries,title=~]
{Convierte una cadena de texto a un número entero.}
\end{tcolorbox}\bigskip 

 

\\{(f) ¿Cual es la diferencia entre argumento y parametro?}

\begin{tcolorbox}[colback=blue!5!white,colframe=blue!75!black,fonttitle=\bfseries,title=~]
 {Un argumento es el valor que se pasa a un parámetro del métdo cuando se le llama.\\
	
	El parámetro es el tipo de dato que va a recibir, proveniente del método, además es este quién define si tiene o no parámetros y de ser el caso, especifica su tipo.}
\end{tcolorbox}\bigskip 


 





\end{document}