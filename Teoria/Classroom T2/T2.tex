\documentclass{article}
\usepackage[top=2cm,bottom=2cm,left=2cm,rigth=2cm,heightrounded]{geometry}
\usepackage[utf8]{inputenc}
\usepackage{graphicx}
\usepackage[spanish]{babel}
\usepackage{schemata}
\usepackage{latexsym}
\usepackage{booktabs}
\usepackage{multirow}


\title{Tarea 2 \\Introducción a Ciencias de la Computación }
\author{Roberto Adrián Bonilla Ruiz}
\date{5 de Marzo 2020}

\begin{document}

\maketitle
%\large
\normalsize\textbf{Basado en la especificación, encuentre las clases, objetos y resposabilidades} 

\centerline\textbf{\rhd \; \textbf{SUSTANTIVOS: }} 
{Cliente, sistema, calculadora, operaciones, universidad, resultado, nùmeros,}\\,\qquad \qquad  \qquad \qquad \qquad { \; \; \; \; estilo, alumno, logo, director.} \\




%\begin{}
     {\rhd \; \textbf{CLASE: }} Calculadora\\
     
     {\rhd \; \textbf{OBJETO: }} Calculadora \\
     
     {\rhd \; \textbf{RESPONSABILIDADES: }} {sumar, restar, multiplicar, dividir, módulo, sumarBinaria,}
     \\   { \qquad \qquad \qquad \qquad \qquad \qquad \qquad \qquad sumarPostfijo, sumarSufijo, ordenar, restarBinaria, restarSufija,} \\ {\qquad \qquad \qquad \qquad \qquad \qquad \qquad \qquad restaPostfija.} \\
     
%\end{}

\newpage
\centering
\large\textbf{Realiza el análisis y diseño, el sistema incluye además las tarjetas de responsabilidades y los diagramas de Warner-Orr.} \\


\begin{table}[h!]
        \centering
         
        \begin{tabular}{ | p{14.9cm} |}
        \hline
            \begin{center} \textbf{CLASE: Calculadora} \end{center} 
            \end{tabular}
            
        \begin{tabular}{| p{2 cm} | p{12.5cm} |}        
        \hline
        % 4 y 9.5 cm
        
        \begin{center} \textbf{Privados} \end{center} & {}
         \\\hline 
         
         
        \begin{center} \textbf{Publicos} \end{center} & 
        \begin{enumerate}
            \item \textbf{sumarBinaria:}{ Recibe dos números enteros, los suma y devuelve su resultado }  
            
         \item  \textbf{restarBinaria:}{Recibe dos números enteros, los resta y devuelve su resultado}  
         
         \item  \textbf{multiplicar: } { Recibe dos números enteros, los multiplica y devuelve su producto}  
         
         \item  \textbf{dividir:} Recibe dos números enteros, los divide y devuelve su resultado
          
         
         
         \item  \textbf{modular:} Recibe dos números enteros, los divide y devuelve su devuelve el residuo de la división 
          
         
         \item  \textbf{sumarPostfijo: } 
        Recibe dos números enteros y realiza la suma de manera postfija.
        
         \item  \textbf{restarPostfijo: }  Recibe dos números enteros y realiza la resta de manera postfija
         
         
         \item  \textbf{sumarSufijo: } Recibe dos números enteros y realiza la suma de manera sufija
         
         \item  \textbf{restarSufijo: } Recibe dos números enteros y realiza la resta de manera sufija 
         
         \item  \textbf{orden: } Recibe tres números enteros,busca el mayor, menor y compara si dos o más números son iguales.
       
        \end{enumerate}
          
         
         
         \\\hline
    \end{tabular}
    \caption{Tarjeta de responsabilidades de la clase Calculadora.}
\end{table}


\\
\\
\\
\vspace{}


\newpage
\centering\textbf{{DIAGRAMA DE WARNER - ORR} \\}{Operaciones Aritméticas y \\Lógicas} \\\


\footnotesize\newcommand\diagram[2]{\schema{\schemabox{#1}}{\schemabox{#2}}}
   \diagram{\boldalert{Operaciones Aritméticas}\\
   {y Lógicas}}
{
\diagram{sumarBinaria}{Recibe a y b números enteros \\ Suma a y b \\ Devuelve resultado}\\
{}\\
{}\\

\diagram{restarBinaria}{Recibe a y b números enteros \\ Resta a y b \\ Devuelve resultado} \\
{}\\
{}\\

\diagram{multiplicar}{Recibe a y b números enteros \\ 
Multiplica a y b \diagram{}{Guardar la multiplicación en \\ la variable resultado} \\  Devuelve resultado} \\
{}\\
{}\\

\diagram{dividir}{Recibe a y b números enteros \\ Divide a y b \\ Devuelve resultado} \\
{}\\
{}\\

\diagram{modulo}{Recibe a y b números enteros \\ 
Divide a y b \diagram{}{Guarda el residuo de la división en \\ la variable resultado} \\  Devuelve resultado} \\
{}\\
{}\\

\diagram{sumarPostfija }{Recibe a y b números enteros \\ Aplica suma postfija a cada nùmero \\ Suma a y b con postfijo \\ Devuelve resultado }\\
{}\\
{}\\

\diagram{restarPostfija }{Recibe a y b números enteros \\ Aplica resta postfija a cada nùmero \\ Resta a y b con postfijo \\ Devuelve resultado }\\
{}\\
{}\\

\diagram{sumarSufija }{Recibe a y b números enteros \\ Aplica suma sufija a cada nùmero \\ Suma a y b con sufijo \\ Devuelve resultado}\\
{}\\
{}\\

\diagram{restarSufija }{Recibe a y b números enteros \\ Aplica resta postfija a cada nùmero \\ Resta a y b con postfijo \\ Devuelve resultado }\\
{}\\
{}\\



\diagram{orden}{Recibe a b y c números enteros \\ Elige el mayor de los tres \\ Devuelve el número\diagram{}{Mayor\\ Menor \\ Compara si es que son iguales o no} \\ { Devuelve resultado}

}

{}\\
{}\\

\diagram{conjunción }{Recibe a y b valores de verdad \\ Realiza la conjuncion\\ Devuelve resultado booleano }\\
{}\\
{}\\

\diagram{disyunción}{Recibe a y b valores de verdad \\ Realiza la disyuncion\\ Devuelve resultado booleano }\\
{}\\
{}\\

}

\end{document}